We now set the following notation such that remains fixed for the remainder of this manuscript
\begin{itemize}
    \setlength\itemsep{1.6em}
    \item Let be a function $f\colon \mathbb{T} \to \mathbb{R}$ and $t\in\mathbb{T}^{\kappa}$ then $f^{\Delta}(t)$
    is delta timescale derivative~\cite{Bohner2001DynamicEO}
    \begin{align*}
        f^{\Delta} (t) = \frac{f(\sigma(t)) - f(t)}{\sigma(t) - t}
    \end{align*}
    where $\sigma(t) - t \neq 0$ and $\sigma(t) > t$ is forward jump operator.

    \item $\dfrac{\partial f(t_1,\ldots,t_n)}{\Delta_i t_i}$ is the delta partial derivative
    of $f\colon \Lambda^n \to \mathbb{R}$ on $n$-dimensional timescale
    $\Lambda^n$ defined via the limit~\cite{bohner2004partial, ahlbrandt2002partial,JACKSON2006391}
    \begin{align*}
        \dfrac{\partial f(t_1,\ldots,t_n)}{\Delta_i t_i} = \lim \limits_{\substack{s_i \to t_i}}
        \frac{
            f(t_1, \ldots, t_{i-1}, \sigma_i(t_i), t_{t+1}, \ldots, t_n)
            - f(t_1, \ldots, t_{i-1}, s_i, t_{t+1}, \ldots, t_n)
        }{\sigma_i(t_i) - s_i}
    \end{align*}
    where $\sigma_i(t_i) > t_i$ and $\sigma_i(t_i) - s_i \neq 0$.

    \item $\qderivative{f(x)}$ is $q-$derivative~\cite{jackson_1909,ernst2000history,ernst2008different,kac2001quantum}
    \begin{align*}
        \qderivative{f(x)} = \frac{f(qx)-f(x)}{qx-x}
    \end{align*}
    where $x\neq 0, \; x\in\mathbb{R}, \; q\in\mathbb{R}$.

    \item $\nqderivative{f(t)}$ is $q-$power derivative~\cite{aldwoah2011power}
    \begin{align*}
        \nqderivative{f(t)} = \frac{f(qt^n) - f(t)}{qt^n - t}
    \end{align*}
    where $qt^n - t \neq 0$ and $n$ is odd positive integer and $0 < q < 1$.

    \item $\qpowerDerivative{f(x)}$ is $q-$power derivative
    \begin{align*}
        \qpowerDerivative{f(x)} = \frac{f(x^q)-f(x)}{x^q-x}
    \end{align*}
    where $x^q \neq x, \; x\in\mathbb{R}, \; q\in\mathbb{R}$.

    \item $\polynomialP{m}{b}{x}$ is $2m+1$-degree polynomial in $x,b$
    \begin{equation}
        \polynomialP{m}{b}{x} = \sum_{k=0}^{b-1} \sum_{r=0}^{m} \coeffA{m}{r} k^r(x-k)^r
        \label{eq:polynomial_p}
    \end{equation}
    where $\coeffA{m}{r}$ is a real coefficient defined recursively, see~\cite{kolosov2016link}.

    \item $\mathbb{Z}$ is an integer timescale such that $\sigma(t) = t+1$ and $\mu(t) = 1$.

    \item $\mathbb{R}$ is a real timescale such that $\sigma(t) = t+\Delta t$ and $\mu(t) = \Delta t, \; \Delta t \to 0$.

    \item $q^\mathbb{R}$ is a quantum timescale such that $\sigma(t) = qt$ and $\mu(t) = qt - t$,
    [page 18~\cite{Bohner2001DynamicEO}].

    \item $\mathbb{R}^q$ is a quantum power timescale such that $\sigma(t) = t^q$ and $\mu(t) = t^q - t$.

    \item $q^{\mathbb{R}^n}$ is a pure quantum power timescale
    such that $\sigma(t) = qt^n > t, \; 0<q<1, \; \mu(t) = qt^n - t$ and $n$ is positive
    odd integer~\cite{aldwoah2011power}.
\end{itemize}
