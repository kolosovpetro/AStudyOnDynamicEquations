We now set the following notation, which remains fixed for the remainder of this paper:
\begin{itemize}
    \setlength\itemsep{1em}
    \item Let be a function $f\colon \mathbb{T} \to \mathbb{R}$ and $t\in\mathbb{T}^{\kappa}$ then $f^{\Delta}(t)$
    is delta time-scale derivative~\cite{Bohner2001DynamicEO} of $f$
    \[
        f^{\Delta} (t) = \frac{f(\sigma(t)) - f(t)}{\mu(t)} = \frac{f(\sigma(t)) - f(t)}{\sigma(t) - t},
    \]
    where $\mu(t) = \sigma(t) - t, \; \mu(t) \neq 0$ and $\sigma(t) > t$ is forward jump operator.

    \item $f^{\Delta_i}_{t_i}(t)$ is delta partial derivative
    of $f\colon \Lambda^n \to \mathbb{R}$ on $n-$dimensional time scale
    $\Lambda^n$~\cite{bohner2004partial, ahlbrandt2002partial,JACKSON2006391},
    defined as a limit
    \[
        f^{\Delta_i}_{t_i}(t) = \lim \limits_{\substack{s_i \to t_i \\ s_i \neq \sigma_i(t_i)}}
        \frac{
            f(t_1, \ldots, t_{i-1}, \sigma_i(t_i), t_{t+1}, \ldots, t_n)
            - f(t_1, \ldots, t_{i-1}, s_i, t_{t+1}, \ldots, t_n)
        }{\sigma_i(t_i) - s_i},
    \]
    where $\sigma_i(t_i) > t_i$ and $\sigma_i(t_i) - s_i \neq 0$.

    \item $\qderivative{f(x)}$ is $q-$derivative~\cite{jackson_1909,ernst2000history,ernst2008different,kac2001quantum}
    \[
        \qderivative{f(x)} = \frac{f(qx)-f(x)}{qx-x},
    \]
    where $x\neq 0, \; x\in\mathbb{R}, \; q\in\mathbb{R}$.

    \item $\nqderivative{f(t)}$ is $q-$power derivative~\cite{aldwoah2011power}
    \[
        \nqderivative{f(t)} = \frac{f(qt^n) - f(t)}{qt^n - t},
    \]
    where $qt^n - t \neq 0$ and $n$ is odd positive integer and $0 < q < 1$.

    \item $\qpowerDerivative{f(x)}$ is $q-$power derivative
    \[
        \qpowerDerivative{f(x)} = \frac{f(x^q)-f(x)}{x^q-x},
    \]
    where $x^q \neq x, \; x\in\mathbb{R}, \; q\in\mathbb{R}$.

    \item $\polynomialP{m}{b}{x}, \; x,b\in\mathbb{R}, \; m\in\mathbb{N}$ is $2m+1-$degree integer-valued polynomial~\cite{kolosov2016link}
    \begin{equation}
        \polynomialP{m}{b}{x} = \sum_{k=0}^{b-1} \sum_{r=0}^{m} \coeffA{m}{r} k^r(x-k)^r,
        \label{eq:polynomial_p}
    \end{equation}
    where $\coeffA{m}{r}, \; m\in\mathbb{N}$ is a real coefficient defined recursively
    \[
        \coeffA{m}{r} =
        \begin{cases}
        (2r+1)
            \binom{2r}{r}, & \text{if } r=m,\\
            (2r+1) \binom{2r}{r} \sum_{d=2r+1}^{m} \coeffA{m}{d} \binom{d}{2r+1} \frac{(-1)^{d-1}}{d-r}
            \bernoulli{2d-2r}, & \text{if } 0 \leq r<m,\\
            0, & \text{if } r<0 \text{ or } r>m,
        \end{cases}
    \]
    where $\bernoulli{t}$ are Bernoulli numbers~\cite{WeissteinBernoulli}.
    It is assumed that $\bernoulli{1}=\frac{1}{2}$.

    \item $\mathbb{Z}$ is an integer time scale such that $\sigma(t) = t+1$ and $\mu(t) = 1$.

    \item $\mathbb{R}$ is a real time scale such that $\sigma(t) = t+\Delta t$ and $\mu(t) = \Delta t, \; \Delta t \to 0$.

    \item $q^\mathbb{R}$ is a quantum time scale such that $\sigma(t) = qt$ and $\mu(t) = qt - t$,
    [page 18~\cite{Bohner2001DynamicEO}].

    \item $\mathbb{R}^q$ is a quantum power time scale such that $\sigma(t) = t^q$ and $\mu(t) = t^q - t$.

    \item $q^{\mathbb{R}^n}$ is a quantum power time scale
    such that $\sigma(t) = qt^n > t, \; 0<q<1, \; \mu(t) = qt^n - t$ and $n$ is positive
    odd integer~\cite{aldwoah2011power}.
\end{itemize}