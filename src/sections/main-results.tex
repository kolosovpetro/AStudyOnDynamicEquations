Timescale derivative of odd-powered polynomial $t^{2m+1}$ may be expressed as follows
\begin{thm}
    \label{main_theorem}
    Let $P(m,b,x)$ be a $2m+1$-degree polynomial in $x,b$.
    Let be a two-dimensional timescale
    $\Lambda^2 = \mathbb{T}_1 \times \mathbb{T}_2 = \{t=(x, b) \colon \; x\in\mathbb{T}_1, \; b\in\mathbb{T}_2 \}$
    such that $\mathbb{T}_1 = \mathbb{T}_2$.
    For every $t\in\mathbb{T}_1$ and $x,b\in \Lambda^2$
    \[
        \frac{\Delta x^{2m+1}}{\Delta x}(t) =
        \frac{\partial P(m,b,x)}{\Delta x} (m, \sigma(t), t) +
        \frac{\partial P(m,b,x)}{\Delta b} (m, t, t)
    \]
    where
    \begin{itemize}
        \setlength\itemsep{1em}
        \item  $\sigma(t) > t$ -- is forward jump operator

        \item $\frac{\partial \polynomialP{m}{b}{x}}{\Delta x} (m, \sigma(t), t)$ --
        is the value of the partial derivative on time scales of
        $\polynomialP{m}{b}{x}$ with respect to the variable $x$ evaluated in point $(x, b)=  (t, \sigma(t))$

        \item $\frac{\partial \polynomialP{m}{b}{x}}{\Delta b} (m, t, t)$ --
        is the value of the partial derivative on time scales of
        $\polynomialP{m}{b}{x}$ with respect to the variable $b$, evaluated at $(x,b) = (t, t)$
    \end{itemize}
\end{thm}
In simpler words, the theorem ~\ref{main_theorem} says
\begin{center}
    \begin{quotation}
        For every odd-powered polynomial $x^{2m+1}$, the derivative on time scales $\frac{\Delta x^{2m+1}}{\Delta x}$
        evaluated in point $t\in\mathbb{T}_1$ equals to partial derivative on time scales of the polynomial
        $\polynomialP{m}{b}{x}$
        with respect to $x$
        evaluated in point
        $(x,b) = (t, \sigma(t))$
        plus the value of the partial derivative on time scales of the polynomial
        $\polynomialP{m}{b}{x}$
        with respect to $b$,
        evaluated in point
        $(x,b)=(t,t)$.
    \end{quotation}
\end{center}

In its extended form the theorem ~\ref{main_theorem} is as follows

\begin{align*}
    \frac{\Delta x^{2m+1}}{\Delta x}(t)
    &= \frac{\partial}{\Delta x} \left( \sum_{k=0}^{b-1} \sum_{r=0}^{m} \coeffA{m}{r} k^r(x-k)^r \right) (m, \sigma(t), t) \\
    &+ \frac{\partial}{\Delta b} \left( \sum_{k=0}^{b-1} \sum_{r=0}^{m} \coeffA{m}{r} k^r(x-k)^r \right) (m, t, t)
\end{align*}
