%! suppress = SuspiciousSectionFormatting
By~\cite[Lemma 3.1]{kolosov2016link}, for every $x\in\mathbb{R}, \; m\in\mathbb{N}$ it is true that
\begin{equation}
    \label{eq:odd_power_identity}
    \polynomialP{m}{x}{x} = x^{2m+1}
\end{equation}

\subsection{Proof of theorem~\ref{main_theorem}}
\label{subsec:proof-of-theorem}
Let be
$x,b\in \Lambda^2 = \mathbb{T}_1 \times \mathbb{T}_2
\colonequals \{t = (x,b) \colon \; x\in\mathbb{T}_1, \; b\in\mathbb{T}_2\}$.
Let be $\mathbb{T}_1 = \mathbb{T}_2$.
Assume that timescale derivative $(x^{2m+1})^{\Delta}$ is
\begin{equation}
    \label{eq:proof1}
    (x^{2m+1})^{\Delta}
    = \lim \limits_{b \to x}
    \lim \limits_{t \to x}
    \frac{\polynomialP{m}{\sigma(b)}{\sigma(x)} - \polynomialP{m}{b}{t}}{\sigma(x) - t}
\end{equation}
where $\sigma(x) > x$ is forward jump operator.
However, equation ~\eqref{eq:proof1} is not a timescale derivative of $\polynomialP{m}{b}{x}$ over $x$
how it might seem because of denominator $\sigma(x) - t$.
Parameter $b$ of $\polynomialP{m}{b}{x}$ is implicitly incremented as well.
Let's try to express nominator of ~\eqref{eq:proof1} in terms of
partial derivative $\pTsDerivative{\polynomialP{m}{b}{x}}{b}$ on timescales.
Let be the following equation
\[
    \polynomialP{m}{\sigma(b)}{x} - \polynomialP{m}{b}{x}
    = \polynomialP{m}{b}{x}^{\Delta}_{b} \cdot \Delta b
\]
Let $t \to x$ in ~\eqref{eq:proof1}.
Then nominator of ~\eqref{eq:proof1} equals to
\[
    \polynomialP{m}{\sigma(b)}{\sigma(x)} - \polynomialP{m}{b}{x}
    = \polynomialP{m}{\sigma(b)}{x} - \polynomialP{m}{b}{x} + A
\]
where $A$ is yet implicit term.
Let's now collapse the terms $f_m (x, b)$ from both sides of above equation, such that
\[
    \polynomialP{m}{\sigma(b)}{\sigma(x)} = \polynomialP{m}{\sigma(b)}{\sigma(x)} + A
\]
Therefore,
\[
    A = \polynomialP{m}{\sigma(b)}{\sigma(x)} - \polynomialP{m}{\sigma(b)}{\sigma(x)}
    = \polynomialP{m}{b}{x}^{\Delta}_{x} (x, \sigma(b)) \cdot \Delta x
\]
Now, let's express the nominator of ~\eqref{eq:proof1} as follows
\begin{align*}
    \polynomialP{m}{\sigma(b)}{\sigma(x)} - \polynomialP{m}{b}{x}
    &= \polynomialP{m}{b}{x}^{\Delta}_{x} (x, \sigma(b)) \cdot \Delta x + \polynomialP{m}{b}{x}^{\Delta}_{b} (x,b) \cdot \Delta b \\
    \polynomialP{m}{\sigma(b)}{\sigma(x)} - \polynomialP{m}{b}{x}
    &= \polynomialP{m}{b}{x}^{\Delta}_{x} (x, \sigma(b)) \cdot (\sigma(x) - x) + \polynomialP{m}{b}{x}^{\Delta}_{b} (x,b) \cdot (\sigma(b) - b)
\end{align*}
We can collapse the terms $(\sigma(x) - x), \; (\sigma(b) - b)$ in above expressions, as $b\to x$.
Therefore,
\begin{align*}
    \frac{\polynomialP{m}{\sigma(x)}{\sigma(x)} - \polynomialP{m}{x}{x}}{\sigma(x) - x}
    = \polynomialP{m}{b}{x}^{\Delta}_{x} (m, \sigma(x), x)
    + \polynomialP{m}{b}{x}^{\Delta}_{b} (m, x, x)
\end{align*}
Finally, by the identity ~\eqref{eq:odd_power_identity} we can express
timescale derivative of $x^{2m+1}, \; x\in \Lambda^2 = \mathbb{T}_1 \times \mathbb{T}_2, \; m\in\mathbb{N}$
as
\begin{equation*}
(x^{2m+1})
    ^{\Delta}(t)= \pTsDerivative{\polynomialP{m}{b}{x}}{x} (m, \sigma(x), x)
    + \pTsDerivative{\polynomialP{m}{b}{x}}{b} (m, x, x)
\end{equation*}

This completes the proof. \qed
