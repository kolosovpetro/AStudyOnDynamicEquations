\begin{cor}
    \label{q_power_derivative_case}
    (Q-power derivative~\cite{aldwoah2011power}.)
    Let be a two-dimensional time scale
    $\Lambda^2 = {\mathbb{R}}^{q} \times {\mathbb{R}}^{q} \colonequals \{t=(x,b) \colon \; b\in {\mathbb{R}}^{q}, \; x\in{\mathbb{R}}^{q} \}$.
    For every $t\in {\mathbb{R}}^{q}, \; (x,b) \in\Lambda^2 = {\mathbb{R}}^{q} \times {\mathbb{R}}^{q}$
    \begin{align*}
        \qpowerDerivative{t^{2m+1}}
        = \pTsDerivative{\polynomialP{m}{b}{x}}{x}(m, \sigma(t), t)
        + \pTsDerivative{\polynomialP{m}{b}{x}}{b}(m, t, t)
    \end{align*}
    where the forward jump operator is defined as $\sigma(t) = t^q, \; q > 1$.
\end{cor}
\begin{examp}
    \label{time_scale_nq_example_1}
    Let be $t\in {\mathbb{R}}^{q}, \; (x,b) \in\Lambda^2 = {\mathbb{R}}^{q} \times {\mathbb{R}}^{q}$, let $m=1$ then
    \begin{align*}
        &\pTsDerivative{\polynomialP{1}{b}{x}}{x} = -3 b + 3 b^2 \\
        &\pTsDerivative{\polynomialP{1}{b}{x}}{b} = 3 b - 2 b^2 + 3 b^q - 2 b^{2 q} - 2 b^{1 + q} - 3 x + 3 b x + 3 b^q x
    \end{align*}
    Evaluating in points yields
    \begin{align*}
        &\pTsDerivative{\polynomialP{1}{b}{x}}{x} (1, \sigma(t), t) = -3 t^q + 3 t^{2 q} \\
        &\pTsDerivative{\polynomialP{1}{b}{x}}{b} (1, t, t) = t^2 + 3 t^q - 2 t^{2 q} + t^{1 + q}
    \end{align*}
    Summing up previously obtained partial time-scale derivatives, we get $q-$power derivative of odd polynomial
    $x^{3}$ evaluated in point $t \in {\mathbb{R}}^{q}$
    \begin{align*}
        \qpowerDerivative{t^{3}}
        = \pTsDerivative{\polynomialP{1}{b}{x}}{x} (1, \sigma(t), t)
        + \pTsDerivative{\polynomialP{1}{b}{x}}{b} (1, t, t)
        = t^2 + t^{2 q} + t^{1 + q}.
    \end{align*}
\end{examp}
\begin{examp}
    \label{time_scale_nq_example_2}
    Let be $t\in {\mathbb{R}}^{q}, \; (x,b) \in \Lambda^2 = {\mathbb{R}}^{q} \times {\mathbb{R}}^{q}$, let $m=2$ then
    \begin{align*}
        \pTsDerivative{\polynomialP{2}{b}{x}}{x}
        &= -15 b^2 + 30 b^3 - 15 b^4 + 5 b x - 15 b^2 x + 10 b^3 x + 5 b x^q - 15 b^2 x^q + 10 b^3 x^q \\
        \pTsDerivative{\polynomialP{2}{b}{x}}{b}
        &= 10 b^2 - 15 b^3 + 6 b^4 + 10 b^{2 q} - 15 b^{3 q} + 6 b^{4 q}
        + 10 b^{1 + q} - 15 b^{2 + q} + 6 b^{3 + q} \\
        &- 15 b^{1 + 2 q} + 6 b^{2 + 2 q} + 6 b^{1 + 3 q} - 15 b x + 30 b^2 x - 15 b^3 x
        - 15 b^q x + 30 b^{2 q} x \\
        &- 15 b^{3 q} x + 30 b^{1 + q} x - 15 b^{2 + q} x - 15 b^{1 + 2 q} x + 5 x^2 - 15 b x^2 + 10 b^2 x^2 \\
        &- 15 b^q x^2 + 10 b^{2 q} x^2 + 10 b^{1 + q} x^2
    \end{align*}
    Evaluation in points yields
    \begin{align*}
        &\pTsDerivative{\polynomialP{2}{b}{x}}{x} (2, \sigma(t), t)
        = -10 t^{2 q} + 15 t^{3q} - 5 t^{4q} + 5 t^{1+q} - 15 t^{1+2q} + 10 t^{1 + 3 q} \\
        &\pTsDerivative{\polynomialP{2}{b}{x}}{b} (2, t, t)
        = t^4 + 10 t^{2 q} - 15 t^{3 q} + 6 t^{4 q} - 5 t^{1 + q} + t^{3 + q}
        + 15 t^{1 + 2 q} + t^{2 + 2 q} - 9 t^{1 + 3 q}
    \end{align*}
    Summing up previously obtained partial time-scale derivatives, we get $q-$power derivative of odd-powered polynomial
    $x^5$ evaluated in point $t\in {\mathbb{R}}^{q}$
    \begin{align*}
        \qpowerDerivative{x^{5}}(t)
        = \pTsDerivative{\polynomialP{2}{b}{x}}{x} (m, \sigma(t), t)
        + \pTsDerivative{\polynomialP{2}{b}{x}}{b} (m, t, t)
        = t^4 + t^{4 q} + t^{3 + q} + t^{2 + 2 q} + t^{1 + 3 q}.
    \end{align*}
\end{examp}
Another polynomial identity, that is exponential sum holds
\begin{cor}
    \label{time_scale_nq_corollary_1}
    For every $t\in {\mathbb{R}}^{q}, \; (x,b) \in \Lambda^2 = {\mathbb{R}}^{q} \times {\mathbb{R}}^{q}, \; t\in\mathbb{R}$
    \begin{align*}
        \lim \limits_{q \to 0} \pTsDerivative{\polynomialP{m}{b}{x}}{b} (m, t, t)
        = \sum_{k=0}^{2m} t^k
    \end{align*}
\end{cor}
