To understand the nature of theorem ~\ref{main_theorem}, let's discuss an example of some popular time scales,
like integer time scale $\mathbb{Z}$, real time scale $\mathbb{R}$, quantum time scale $q^{\mathbb{R}}$,
quantum-power time scale $\mathbb{R}^q$.
We use the principle \textit{Divide et Impera !} in order to understand entire behavior of theorem~\ref{main_theorem}.

\subsection{Time scale of integers $\mathbb{T} = \mathbb{Z} \times \mathbb{Z}$} \label{subsec:time_scale_z}
\begin{cor}
    \label{finite_difference_case}
    (Finite difference.)
    Let be a two-dimensional timescale
    $\Lambda^2 = \mathbb{Z} \times \mathbb{Z}$.
    For every $t\in\mathbb{Z}$ and $x,b\in \Lambda^2$
    \begin{align*}
        \finiteDifference{x^{2m+1}} (t)
        = \pTsDerivative{\polynomialP{m}{b}{x}}{x} (m, \sigma(t), t)
        + \pTsDerivative{\polynomialP{m}{b}{x}}{b}(m, \sigma(t), t)
    \end{align*}
    where $\sigma(t)$ is the forward jump operator defined as $\sigma(t) = t+1$.
\end{cor}
\begin{examp}
    \label{time_scale_z_example_1}
    Let be $t \in \mathbb{Z}, \; x,b \in \Lambda^2 = \mathbb{Z} \times \mathbb{Z}, \; m\in\mathbb{N}$ and let $m=1$, then
    \begin{align*}
        &\pTsDerivative{\polynomialP{1}{b}{x}}{x}                = -3 b + 3 b^2 \\
        &\pTsDerivative{\polynomialP{1}{b}{x}}{b}                = 1 - 6 b^2 + 6 b x
%        &\pTsDerivative{\polynomialP{1}{b}{x}}{x} (t, \sigma(t)) = 3 t + 3 t^2 \\
%        &\pTsDerivative{\polynomialP{1}{b}{x}}{b}(t,t)           = 1
    \end{align*}
    Evaluating in points yields
    \begin{align*}
%        &\pTsDerivative{\polynomialP{1}{b}{x}}{x}                = -3 b + 3 b^2 \\
%        &\pTsDerivative{\polynomialP{1}{b}{x}}{b}                = 1 - 6 b^2 + 6 b x \\
        &\pTsDerivative{\polynomialP{1}{b}{x}}{x} (t, \sigma(t)) = 3 t + 3 t^2 \\
        &\pTsDerivative{\polynomialP{1}{b}{x}}{b}(t,t)           = 1
    \end{align*}
    Summing up previously obtained partial timescale derivatives, we get ordinary finite difference of odd-powered polynomial
    $x^{3}$ evaluated in point $ t\in\mathbb{Z}, \; x,b\in\Lambda^2 = \mathbb{Z} \times \mathbb{Z}$
    \begin{align*}
        \finiteDifference{x^{3}}(t)
        = \pTsDerivative{\polynomialP{1}{b}{x}}{x} (t, \sigma(t))
        + \pTsDerivative{\polynomialP{1}{b}{x}}{b}(t,t)
        = 3 t + 3 t^2 + 1
    \end{align*}
\end{examp}
\begin{examp}
    \label{time_scale_z_example_2}
    Let be $t\in\mathbb{Z}, \;x,b\in\Lambda^2 = \mathbb{Z} \times \mathbb{Z}, \; m\in\mathbb{N}$
    and let $m=2$, then
    \begin{align*}
        &\pTsDerivative{\polynomialP{2}{b}{x}}{x}                = 5 b - 30 b^2 + 40 b^3 - 15 b^4 + 10 b x - 30 b^2 x + 20 b^3 x \\
        &\pTsDerivative{\polynomialP{2}{b}{x}}{b}                = 1 + 30 b^4 - 60 b^3 x + 30 b^2 x^2
%        &\pTsDerivative{\polynomialP{2}{b}{x}}{x} (t, \sigma(t)) = 5 t + 10 t^2 + 10 t^3 + 5 t^4 \\
%        &\pTsDerivative{\polynomialP{2}{b}{x}}{b} (t, t)         = 1
    \end{align*}
    Evaluating in points yields
    \begin{align*}
        &\pTsDerivative{\polynomialP{2}{b}{x}}{x} (t, \sigma(t)) = 5 t + 10 t^2 + 10 t^3 + 5 t^4 \\
        &\pTsDerivative{\polynomialP{2}{b}{x}}{b} (t, t)         = 1
    \end{align*}
    Summing up previously obtained partial time-scale derivatives, we get time ordinary finite difference of odd-powered
    polynomial $x^{2m+1}$ and $t\in\mathbb{Z}, \;x,b\in\Lambda^2 = \mathbb{Z} \times \mathbb{Z}, \; m\in\mathbb{N}$
    \begin{align*}
        \finiteDifference{x^{5}} (t)
        = \pTsDerivative{\polynomialP{2}{b}{x}}{x} (t, \sigma(t))
        + \pTsDerivative{\polynomialP{2}{b}{x}}{b} (t, t)
        = 1 + 5 t + 10 t^2 + 10 t^3 + 5 t^4
    \end{align*}
\end{examp}
\begin{cor}
    \label{time_scale_z_corollary_1}
    For every $t\in\mathbb{Z}, \; x,b\in\Lambda^2 = \mathbb{Z} \times \mathbb{Z}, \; m\in\mathbb{N}$
    \[
        \pTsDerivative{\polynomialP{m}{b}{x}}{x} \bigg |_{x = t, \; b = \sigma(t)}
        = \sum_{r=1}^{2m} \binom{2m+1}{r} t^{r}
    \]
\end{cor}
\begin{cor}
    \label{time_scale_z_corollary_2}
    For every $t\in\mathbb{Z}, \; x,b\in\Lambda^2 = \mathbb{Z} \times \mathbb{Z}, \; m\in\mathbb{N}$
    \[
        \pTsDerivative{\polynomialP{m}{b}{x}}{b} \bigg |_{x = t, \; b = t} = 1
    \]
\end{cor}


\subsection{Time scale of real numbers $\mathbb{T} = \mathbb{R} \times \mathbb{R}$} \label{subsec:time_scale_r}
\begin{cor}
    \label{derivative_case}
    (Classical derivative.)
    Let be a two-dimensional time scale
    $\Lambda^2 = \mathbb{R} \times \mathbb{R} \colonequals \{t=(x, b) \colon \; x\in \mathbb{R}, \; b\in\mathbb{R} \}$.
    For every $t\in\mathbb{R}, \; x,b\in \Lambda^2 = \mathbb{R} \times \mathbb{R}, \; m\in\mathbb{N}$
    \[
        \derivative{t^{2m+1}}{t}
        = \pderivative{\polynomialP{m}{b}{x}}{x} \bigg |_{x = t, \; b = \sigma(t)}
        + \pderivative{\polynomialP{m}{b}{x}}{b} \bigg |_{x = t, \; b = t},
    \]
    where $\sigma(t) = t + \Delta t, \; \Delta t \to 0.$
\end{cor}
\begin{examp}
    Let be $t\in\mathbb{R}, \; x,b\in \Lambda^2 = \mathbb{R} \times \mathbb{R}, \; m\in\mathbb{N}$ and let $m=1$, then
    \begin{align*}
        \pderivative{\polynomialP{1}{b}{x}}{x} &= -3 b + 3 b^2 \\
        \pderivative{\polynomialP{1}{b}{x}}{b} &= 6 b - 6 b^2 - 3 x + 6 b x
    \end{align*}
    \begin{align*}
        &\pderivative{\polynomialP{1}{b}{x}}{x}(t, \sigma(t)) = -3t + 3t^2 \\
        &\pderivative{\polynomialP{1}{b}{x}}{b} (t, t) = 3t
    \end{align*}
    Summing up previously obtained partial time-scale derivatives, we get classical derivative of odd polynomial
    $t^{2m+1}, t\in\mathbb{R}, \; x\in \Lambda^2 = \mathbb{R} \times \mathbb{R}, \; m\in\mathbb{N}$
    \[
        \derivative{t^3}{t}
        = \pderivative{\polynomialP{1}{b}{x}}{x}(t, \sigma(t)) + \pderivative{\polynomialP{1}{b}{x}}{b}(t, t)
        = 3t^2.
    \]
\end{examp}
\begin{examp}
    Let be $t\in\mathbb{R}, \; x,b\in \Lambda^2 = \mathbb{R} \times \mathbb{R}, \; m\in\mathbb{N}$ and let $m=2$, then
    \begin{align*}
        &\pderivative{\polynomialP{2}{b}{x}}{x} = -15 b^2 + 30 b^3 - 15 b^4 + 10 b x - 30 b^2 x + 20 b^3 x, \\
        &\pderivative{\polynomialP{2}{b}{x}}{b} = 30 b^2 - 60 b^3 + 30 b^4 - 30 b x + 90 b^2 x - 60 b^3 x + 5 x^2 - 30 b x^2
        + 30 b^2 x^2 \\
        &\pderivative{\polynomialP{2}{b}{x}}{x}(t, \sigma(t)) = -5 t^2 + 5 t^4 \\
        &\pderivative{\polynomialP{2}{b}{x}}{b} (t, t) = 5 t^2
    \end{align*}
    Summing up previously obtained partial time-scale derivatives, we get classical derivative of an odd polynomial
    $t^{2m+1}, \; t\in\mathbb{R}, \; x\in \Lambda^2 = \mathbb{R} \times \mathbb{R}, \; m\in\mathbb{N}$
    \[
        \derivative{t^5}{t}
        = \pderivative{\polynomialP{2}{b}{x}}{x}(t, \sigma(t)) + \pderivative{\polynomialP{2}{b}{x}}{b}(t, t)
        = 5t^4.
    \]
\end{examp}

\subsection{Quantum time scale $\mathbb{T} = q^\mathbb{R} \times q^\mathbb{R}$} \label{subsec:time_scale_qn}
\begin{cor}
    \label{q_derivative_case}
    (Q-derivative~\cite{jackson_1909}.)
    Let be a two-dimensional time scale
    $\Lambda^2 = q^{\mathbb{R}} \times q^{\mathbb{R}} \colonequals \{t=(x,b) \colon \; x\in q^{\mathbb{R}}, \; b \in q^{\mathbb{R}} \}$.
    For every $t\in q^{\mathbb{R}}$ and $(x,b)\in \Lambda^2$
    \begin{align*}
        \qderivative{x^{2m+1}}(t)
        = \pTsDerivative{\polynomialP{m}{b}{x}}{x} (m, \sigma(t), t)
        + \pTsDerivative{\polynomialP{m}{b}{x}}{b} (m, t, t)
    \end{align*}
    where $\sigma(t) = qt, \; q > 1$.
\end{cor}
\begin{examp}
    \label{time_scale_qn_example_1}
    Let be $t\in q^{\mathbb{R}}, \; x,b\in \Lambda^2 = q^{\mathbb{R}} \times q^{\mathbb{R}}$, let $m=1$ then
    \begin{align*}
        &\pTsDerivative{\polynomialP{1}{b}{x}}{x} = -3 b + 3 b^2 \\
        &\pTsDerivative{\polynomialP{1}{b}{x}}{b} = 3 b - 2 b^2 + 3 b q - 2 b^2 q - 2 b^2 q^2 - 3 x + 3 b x + 3 b q x
    \end{align*}
    Evaluating in points yields
    \begin{align*}
        &\pTsDerivative{\polynomialP{1}{b}{x}}{x} (m, \sigma(t), t) = -3 q t + 3 q^2 t^2 \\
        &\pTsDerivative{\polynomialP{1}{b}{x}}{b} (m, t, t) = 3 q t + t^2 + q t^2 - 2 q^2 t^2
    \end{align*}
    Summing up previously obtained partial time-scale derivatives, we get the $q$-derivative of odd-powered polynomial
    $x^{3}$ evaluated in point $t\in q^{\mathbb{R}}$
    \begin{align*}
        \qderivative{x^{3}}(t)
        = \pTsDerivative{\polynomialP{1}{b}{x}}{x} (m, \sigma(t), t)
        + \pTsDerivative{\polynomialP{1}{b}{x}}{b} (m, t, t)
        = t^2 + q t^2 + q^2 t^2.
    \end{align*}
\end{examp}
For every $t\in q^{\mathbb{R}}, \; (x,b) \in \Lambda^2 = q^{\mathbb{R}} \times q^{\mathbb{R}}$
the following polynomial identity holds as $q$ tends to zero
\begin{align*}
    \lim \limits_{q \to 0} \pTsDerivative{\polynomialP{1}{b}{x}}{b} (1, t, t) = t^2
\end{align*}
However, it would be generalized as follows
\begin{cor}
    \label{time_scale_qn_corollary_1}
    For every $t\in q^{\mathbb{R}}, \; (x,b) \in \Lambda^2 = q^{\mathbb{R}} \times q^{\mathbb{R}}$
    \begin{align*}
        \lim \limits_{q \to 0} \pTsDerivative{\polynomialP{m}{b}{x}}{b} (m, t, t) = t^{2m}.
    \end{align*}
\end{cor}
\begin{examp}
    \label{time_scale_qn_example_2}
    Let be $t\in q^{\mathbb{R}}, \; (x,b) \in \Lambda^2 = q^{\mathbb{R}} \times q^{\mathbb{R}}$, let $m=2$ then
    \begin{align*}
        \pTsDerivative{\polynomialP{2}{b}{x}}{x}
        &= -15 b^2 + 30 b^3 - 15 b^4 + 5 b x - 15 b^2 x + 10 b^3 x + 5 b q x - 15 b^2 q x + 10 b^3 q x \\
        \pTsDerivative{\polynomialP{2}{b}{x}}{b}
        &= 10 b^2 - 15 b^3 + 6 b^4 + 10 b^2 q - 15 b^3 q + 6 b^4 q
        + 10 b^2 q^2 - 15 b^3 q^2 + 6 b^4 q^2 - 15 b^3 q^3 \\
        &+ 6 b^4 q^3 + 6 b^4 q^4 - 15 b x + 30 b^2 x - 15 b^3 x - 15 b q x + 30 b^2 q x
        - 15 b^3 q x + 30 b^2 q^2 x \\
        &- 15 b^3 q^2 x - 15 b^3 q^3 x + 5 x^2 - 15 b x^2 + 10 b^2 x^2 - 15 b q x^2 + 10 b^2 q x^2 + 10 b^2 q^2 x^2
    \end{align*}
    Evaluating in points yields
    \begin{align*}
        &\pTsDerivative{\polynomialP{2}{b}{x}}{x} (2, \sigma(t), t)
        = 5 q t^2 - 10 q^2 t^2 - 15 q^2 t^3 + 15 q^3 t^3 + 10 q^3 t^4 - 5 q^4 t^4 \\
        &\pTsDerivative{\polynomialP{2}{b}{x}}{b} (2, t, t)
        = -5 q t^2 + 10 q^2 t^2 + 15 q^2 t^3 - 15 q^3 t^3 + t^4 + q t^4 + q^2 t^4 - 9 q^3 t^4 + 6 q^4 t^4
    \end{align*}
    Summing up previously obtained partial time-scale derivatives, we get the $q-$derivative of odd polynomial
    $x^{5}$ evaluated in point $t\in q^{\mathbb{R}}$
    \begin{align*}
        \qderivative{t^{5}}
        = \pTsDerivative{\polynomialP{2}{b}{x}}{x} (2, \sigma(t), t)
        + \pTsDerivative{\polynomialP{2}{b}{x}}{b} (2, t, t)
        = t^4 + q t^4 + q^2 t^4 + q^3 t^4 + q^4 t^4.
    \end{align*}
\end{examp}


\subsection{Quantum power time scale $\mathbb{T} = \mathbb{R}^q \times \mathbb{R}^q$} \label{subsec:time_scale_nq}
\begin{cor}
    \label{q_power_derivative_case}
    (Q-power derivative~\cite{aldwoah2011power}.)
    Let be a two-dimensional time scale
    $\Lambda^2 = {\mathbb{R}}^{q} \times {\mathbb{R}}^{q} \colonequals \{t=(x,b) \colon \; b\in {\mathbb{R}}^{q}, \; x\in{\mathbb{R}}^{q} \}$.
    For every $t\in {\mathbb{R}}^{q}, \; (x,b) \in\Lambda^2 = {\mathbb{R}}^{q} \times {\mathbb{R}}^{q}$
    \begin{align*}
        \qpowerDerivative{t^{2m+1}}
        = \pTsDerivative{\polynomialP{m}{b}{x}}{x}(m, \sigma(t), t)
        + \pTsDerivative{\polynomialP{m}{b}{x}}{b}(m, t, t)
    \end{align*}
    where the forward jump operator is defined as $\sigma(t) = t^q, \; q > 1$.
\end{cor}
\begin{examp}
    \label{time_scale_nq_example_1}
    Let be $t\in {\mathbb{R}}^{q}, \; (x,b) \in\Lambda^2 = {\mathbb{R}}^{q} \times {\mathbb{R}}^{q}$, let $m=1$ then
    \begin{align*}
        &\pTsDerivative{\polynomialP{1}{b}{x}}{x} = -3 b + 3 b^2 \\
        &\pTsDerivative{\polynomialP{1}{b}{x}}{b} = 3 b - 2 b^2 + 3 b^q - 2 b^{2 q} - 2 b^{1 + q} - 3 x + 3 b x + 3 b^q x
    \end{align*}
    Evaluating in points yields
    \begin{align*}
        &\pTsDerivative{\polynomialP{1}{b}{x}}{x} (1, \sigma(t), t) = -3 t^q + 3 t^{2 q} \\
        &\pTsDerivative{\polynomialP{1}{b}{x}}{b} (1, t, t) = t^2 + 3 t^q - 2 t^{2 q} + t^{1 + q}
    \end{align*}
    Summing up previously obtained partial time-scale derivatives, we get $q-$power derivative of odd polynomial
    $x^{3}$ evaluated in point $t \in {\mathbb{R}}^{q}$
    \begin{align*}
        \qpowerDerivative{t^{3}}
        = \pTsDerivative{\polynomialP{1}{b}{x}}{x} (1, \sigma(t), t)
        + \pTsDerivative{\polynomialP{1}{b}{x}}{b} (1, t, t)
        = t^2 + t^{2 q} + t^{1 + q}.
    \end{align*}
\end{examp}
\begin{examp}
    \label{time_scale_nq_example_2}
    Let be $t\in {\mathbb{R}}^{q}, \; (x,b) \in \Lambda^2 = {\mathbb{R}}^{q} \times {\mathbb{R}}^{q}$, let $m=2$ then
    \begin{align*}
        \pTsDerivative{\polynomialP{2}{b}{x}}{x}
        &= -15 b^2 + 30 b^3 - 15 b^4 + 5 b x - 15 b^2 x + 10 b^3 x + 5 b x^q - 15 b^2 x^q + 10 b^3 x^q \\
        \pTsDerivative{\polynomialP{2}{b}{x}}{b}
        &= 10 b^2 - 15 b^3 + 6 b^4 + 10 b^{2 q} - 15 b^{3 q} + 6 b^{4 q}
        + 10 b^{1 + q} - 15 b^{2 + q} + 6 b^{3 + q} \\
        &- 15 b^{1 + 2 q} + 6 b^{2 + 2 q} + 6 b^{1 + 3 q} - 15 b x + 30 b^2 x - 15 b^3 x
        - 15 b^q x + 30 b^{2 q} x \\
        &- 15 b^{3 q} x + 30 b^{1 + q} x - 15 b^{2 + q} x - 15 b^{1 + 2 q} x + 5 x^2 - 15 b x^2 + 10 b^2 x^2 \\
        &- 15 b^q x^2 + 10 b^{2 q} x^2 + 10 b^{1 + q} x^2
    \end{align*}
    Evaluation in points yields
    \begin{align*}
        &\pTsDerivative{\polynomialP{2}{b}{x}}{x} (2, \sigma(t), t)
        = -10 t^{2 q} + 15 t^{3q} - 5 t^{4q} + 5 t^{1+q} - 15 t^{1+2q} + 10 t^{1 + 3 q} \\
        &\pTsDerivative{\polynomialP{2}{b}{x}}{b} (2, t, t)
        = t^4 + 10 t^{2 q} - 15 t^{3 q} + 6 t^{4 q} - 5 t^{1 + q} + t^{3 + q}
        + 15 t^{1 + 2 q} + t^{2 + 2 q} - 9 t^{1 + 3 q}
    \end{align*}
    Summing up previously obtained partial time-scale derivatives, we get $q-$power derivative of odd-powered polynomial
    $x^5$ evaluated in point $t\in {\mathbb{R}}^{q}$
    \begin{align*}
        \qpowerDerivative{x^{5}}(t)
        = \pTsDerivative{\polynomialP{2}{b}{x}}{x} (m, \sigma(t), t)
        + \pTsDerivative{\polynomialP{2}{b}{x}}{b} (m, t, t)
        = t^4 + t^{4 q} + t^{3 + q} + t^{2 + 2 q} + t^{1 + 3 q}.
    \end{align*}
\end{examp}
Another polynomial identity, that is exponential sum holds
\begin{cor}
    \label{time_scale_nq_corollary_1}
    For every $t\in {\mathbb{R}}^{q}, \; (x,b) \in \Lambda^2 = {\mathbb{R}}^{q} \times {\mathbb{R}}^{q}, \; t\in\mathbb{R}$
    \begin{align*}
        \lim \limits_{q \to 0} \pTsDerivative{\polynomialP{m}{b}{x}}{b} (m, t, t)
        = \sum_{k=0}^{2m} t^k
    \end{align*}
\end{cor}


\subsection{Pure quantum power time scale $\mathbb{T} = q^{\mathbb{R}^n} \times q^{\mathbb{R}^n}$} \label{subsec:pure_quantum_power}
In this subsection we discuss a pure quantum power time scale $q^{\mathbb{R}^j}$ provided by Aldwoah, Malinowska
and Torres in~\cite{aldwoah2011power}, among with the $q-$power derivative operator $\nqderivative{f(t)}$ defined by
\begin{align*}
    \nqderivative{f(t)} = \frac{f(qt^n) - f(t)}{qt^n - t},
\end{align*}
where $n$ is odd positive integer and $0 < q < 1$.
\begin{cor}
(Quantum power derivative~\cite{aldwoah2011power}.)
    Let be a two-dimensional time scale
    $\Lambda^2 = q^{\mathbb{R}^j} \times q^{\mathbb{R}^j}
    \colonequals \{t=(x,b) \colon \; b \in q^{\mathbb{R}^j}, \; x \in q^{\mathbb{R}^j} \}$.
    For every $t\in q^{\mathbb{R}^j}, \; x,b\in\Lambda^2 = q^{\mathbb{R}^j} \times q^{\mathbb{R}^j}, \; m\in\mathbb{N}$
    \[
        \nqderivative{x^{2m+1}} (t)
        = \pTsDerivative{\polynomialP{m}{b}{x}}{x} (m, \sigma(t), t)
        + \pTsDerivative{\polynomialP{m}{b}{x}}{b} (m, t, t)
    \]
    where $\sigma(t) = qt^n, \; \sigma(t) > t$.
\end{cor}
\begin{examp}
    \label{time_scale_pure_quantum_power_example_1}
    Let be $t\in q^{\mathbb{R}^j}, \; x,b\in\Lambda^2 = q^{\mathbb{R}^j} \times q^{\mathbb{R}^j}, \; m\in\mathbb{N}$
    and let $m=1$, then
    \begin{align*}
        &\pTsDerivative{\polynomialP{1}{b}{x}}{x} = -3 b + 3 b^2 \\
        &\pTsDerivative{\polynomialP{1}{b}{x}}{b} = 3 b - 2 b^2 + 3 b^j q - 2 b^{1 + j} q - 2 b^{2 j} q^2 - 3 x + 3 b x + 3 b^j q x
%        &\pTsDerivative{\polynomialP{1}{b}{x}}{x}(t, \sigma(t)) = -3 q t^j + 3 q^2 t^{2 j} \\
%        &\pTsDerivative{\polynomialP{1}{b}{x}}{b}(t, t) = t^2 + 3 q t^j - 2 q^2 t^{2 j} + q t^{1 + j}
    \end{align*}
    Evaluating in points yields
    \begin{align*}
        &\pTsDerivative{\polynomialP{1}{b}{x}}{x}(1, \sigma(t), t) = -3 q t^j + 3 q^2 t^{2 j} \\
        &\pTsDerivative{\polynomialP{1}{b}{x}}{b}(1, t, t) = t^2 + 3 q t^j - 2 q^2 t^{2 j} + q t^{1 + j}
    \end{align*}
    Summing up previously obtained partial timescale derivatives, we get $q-$power derivative of odd polynomial
    $t^{2m+1}, \; t\in q^{\mathbb{R}^j}, \; x,b\in\Lambda^2 = q^{\mathbb{R}^j} \times q^{\mathbb{R}^j}, \; m\in\mathbb{N}$
    \begin{align*}
        \nqderivative{t^{3}}
        = \pTsDerivative{\polynomialP{1}{b}{x}}{x} (1, \sigma(t), t)
        + \pTsDerivative{\polynomialP{1}{b}{x}}{b} (1, t, t)
        = t^2 + q^2 t^{2 j} + q t^{1 + j}.
    \end{align*}
\end{examp}

Another polynomial identity, that is exponential sum holds
\begin{cor}
    \label{time_scale_pure_quantum_power_corollary_1}
    For every $t\in q^{\mathbb{R}^j}, \; x,b\in\Lambda^2 = q^{\mathbb{R}^j} \times q^{\mathbb{R}^j}, \; t\in\mathbb{R}, \; m\in\mathbb{N}$
    \begin{align*}
        \lim \limits_{j \to 0} \lim \limits_{q \to 1} \pTsDerivative{\polynomialP{m}{b}{x}}{b} (m, t, t)
        = \sum_{k=0}^{2m} t^k
    \end{align*}
\end{cor}

An identity in even polynomials holds too
\begin{cor}
    \label{time_scale_pure_quantum_power_corollary_2}
    For every $t\in q^{\mathbb{R}^j}, \; x,b\in\Lambda^2 = q^{\mathbb{R}^j} \times q^{\mathbb{R}^j}, \; t\in\mathbb{R}, \; m\in\mathbb{N}$
    \begin{align*}
        \lim \limits_{j \to 0} \lim \limits_{q \to 0} \pTsDerivative{\polynomialP{m}{b}{x}}{b} (m, t, t)
        = t^{2m}
    \end{align*}
\end{cor}

\begin{examp}
    \label{time_scale_pure_quantum_power_example_2}
    Let be $t\in q^{\mathbb{R}^j}, \; x,b\in\Lambda^2 = q^{\mathbb{R}^j} \times q^{\mathbb{R}^j}, \; m\in\mathbb{N}$ and let $m=2$, then
    \begin{align*}
        \pTsDerivative{\polynomialP{2}{b}{x}}{x}
        &= -15 b^2 + 30 b^3 - 15 b^4 + 5 b x - 15 b^2 x + 10 b^3 x + 5 b q x^j - 15 b^2 q x^j + 10 b^3 q x^j \\
        \pTsDerivative{\polynomialP{2}{b}{x}}{b}
        &= 10 b^2 - 15 b^3 + 6 b^4 + 10 b^{1 + j} q - 15 b^{2 + j} q + 6 b^{3 + j} q + 10 b^{2 j} q^2 - 15 b^{1 + 2 j} q^2 \\
        &+ 6 b^{2 + 2 j} q^2 - 15 b^{3 j} q^3 + 6 b^{1 + 3 j} q^3 + 6 b^{4 j} q^4 - 15 b x + 30 b^2 x - 15 b^3 x - 15 b^j q x \\
        &+ 30 b^{1 + j} q x - 15 b^{2 + j} q x + 30 b^{2 j} q^2 x - 15 b^{1 + 2 j} q^2 x - 15 b^{3 j} q^3 x + 5 x^2 - 15 b x^2 \\
        &+ 10 b^2 x^2 - 15 b^j q x^2 + 10 b^{1 + j} q x^2 + 10 b^{2 j} q^2 x^2
    \end{align*}
    Evaluation in points yields
    \begin{align*}
        &\pTsDerivative{\polynomialP{2}{b}{x}}{x} (2, \sigma(t), t)
        = -10 q^2 t^{2 j} + 15 q^3 t^{3 j} - 5 q^4 t^{4 j} + 5 q t^{1 + j} - 15 q^2 t^{1 + 2 j} + 10 q^3 t^{1 + 3 j} \\
        &\pTsDerivative{\polynomialP{2}{b}{x}}{b} (2, t, t)
        = t^4 + 10 q^2 t^{2 j} - 15 q^3 t^{3 j} + 6 q^4 t^{4 j} - 5 q t^{1 + j} + q t^{3 + j} + 15 q^2 t^{1 + 2 j}
        + q^2 t^{2 + 2 j} - 9 q^3 t^{1 + 3 j}
    \end{align*}
    Summing up previously obtained partial timescale derivatives, we $q-$power derivative of odd polynomial
    $t^{2m+1}, \; t\in q^{\mathbb{R}^j}, \; x,b\in\Lambda^2 = q^{\mathbb{R}^j} \times q^{\mathbb{R}^j}, \; m\in\mathbb{N}$
    \begin{align*}
        \nqderivative{t^{5}}
        = \pTsDerivative{\polynomialP{1}{b}{x}}{x} (2, \sigma(t), t)
        + \pTsDerivative{\polynomialP{1}{b}{x}}{b} (2, t, t)
        = t^4 + q^4 t^{4 j} + q t^{3 + j} + q^2 t^{2 + 2 j} + q^3 t^{1 + 3 j}
    \end{align*}
\end{examp}
