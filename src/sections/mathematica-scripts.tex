To fulfill our study, we attach here a link to the set of \emph{Mathematica} programs, designed to verify the results of current manuscript.
To reach these programs follow the link~\cite{kolosov2022mathematica}.
To reproduce results, proceed as follows:
\begin{itemize}
    \setlength\itemsep{1em}
    \item Time scale of integers $\mathbb{T} = \mathbb{Z} \times \mathbb{Z}$:
    \begin{itemize}
        \setlength\itemsep{0.5em}
        \item Example~\ref{time_scale_z_example_1}:
        Execute the commands of Mathematica package
        \begin{itemize}
            \item Set \texttt{sigma[x\_] := x + 1} in Mathematica package and execute definition
            \item \texttt{timeScaleDerivativeX[1, x, b]} which produces $-3 b + 3 b^2$.
            \item \texttt{Expand[timeScaleDerivativeX[1, t, sigma[t]]]} which produces $3 t + 3 t^2$.
            \item \texttt{timeScaleDerivativeB[1, x, b]} which produces $1 - 6 b^2 + 6 b x$.
            \item \texttt{timeScaleDerivativeB[1, t, t]} which produces $1$.
            \item \texttt{mainTheorem[1]} which produces $1 + 3 t + 3 t^2$.
        \end{itemize}
        \item Example~\ref{time_scale_z_example_2}:
        Execute the commands of Mathematica package
        \begin{itemize}
            \item Set \texttt{sigma[x\_] := x + 1} in Mathematica package and execute definition
            \item \texttt{timeScaleDerivativeX[2, x, b]} which produces
            $5 b - 30 b^2 + 40 b^3 - 15 b^4 + 10 b x - 30 b^2 x + 20 b^3 x$.
            \item \texttt{Expand[timeScaleDerivativeX[2, t, sigma[t]]]} which produces $5 t + 10 t^2 + 10 t^3 + 5 t^4$.
            \item \texttt{timeScaleDerivativeB[2, x, b]} which produces $1 + 30 b^4 - 60 b^3 x + 30 b^2 x^2$.
            \item \texttt{timeScaleDerivativeB[2, t, t]} which produces $1$.
            \item \texttt{mainTheorem[2]} which produces $1 + 5 t + 10 t^2 + 10 t^3 + 5 t^4$.
        \end{itemize}
    \end{itemize}
    \item Time scale of real numbers $\mathbb{T} = \mathbb{R} \times \mathbb{R}$:
    \begin{itemize}
        \item Example~\ref{time_scale_r_example_1}:
        Execute the commands of Mathematica package
        \begin{itemize}
            \item Set \texttt{sigma[x\_] := x + Global`dx} in Mathematica package and execute definition
            \item Execute \texttt{timeScaleDerivativeX[1, x, b]} which produces $-3 b + 3 b^2$.
            \item Execute \texttt{Limit[Expand[timeScaleDerivativeB[1, x, b]], dx -> 0]}
            which produces $6 b - 6 b^2 - 3 x + 6 b x$.
            \item Execute \texttt{timeScaleDerivativeX[1, t, t]} which produces $-3 t + 3 t^2$.
            \item Execute \texttt{Limit[Expand[timeScaleDerivativeB[1, t, t]], dx -> 0]} which produces $3t$.
            \item Execute \texttt{Limit[mainTheorem[1], dx -> 0]} which produces $3t^2$.
        \end{itemize}
        \item Example~\ref{time_scale_r_example_2}:
        Execute the commands of Mathematica package
        \begin{itemize}
            \item Set \texttt{sigma[x\_] := x + Global`dx} in Mathematica package and execute definition
            \item Execute \texttt{Limit[Expand[timeScaleDerivativeX[2, x, b]], dx -> 0]}
            which produces $-15 b^2 + 30 b^3 - 15 b^4 + 10 b x - 30 b^2 x + 20 b^3 x$.
            \item Execute \texttt{Limit[Expand[timeScaleDerivativeB[2, x, b]], dx -> 0]}
            which produces $30 b^2 - 60 b^3 + 30 b^4 - 30 b x + 90 b^2 x - 60 b^3 x + 5 x^2 -
            30 b x^2 + 30 b^2 x^2$.
            \item Execute \texttt{Limit[Expand[timeScaleDerivativeX[2, t, sigma[t]]], dx -> 0]} which produces $-5 t^2 + 5 t^4$.
            \item Execute \texttt{Limit[Expand[timeScaleDerivativeB[2, t, t]], dx -> 0]} which produces $5t^2$.
            \item Execute \texttt{Limit[mainTheorem[2], dx -> 0]} which produces $5t^4$.
        \end{itemize}
    \end{itemize}
    \item Quantum time scale $\mathbb{T} = q^\mathbb{R} \times q^\mathbb{R}$:
    \begin{itemize}
        \item Example~\ref{time_scale_qn_example_1}:
        \item Example~\ref{time_scale_qn_example_2}:
        \item Corollary~\ref{time_scale_qn_corollary_1}:
    \end{itemize}
    \item Quantum power time scale $\mathbb{T} = \mathbb{R}^q \times \mathbb{R}^q$:
    \begin{itemize}
        \item Example~\ref{time_scale_nq_example_1}:
        \item Example~\ref{time_scale_nq_example_2}:
        \item Corollary~\ref{time_scale_nq_corollary_1}:
    \end{itemize}
    \item Pure quantum power time scale $\mathbb{T} = q^{\mathbb{R}^n} \times q^{\mathbb{R}^n}$:
    \begin{itemize}
        \item Example~\ref{time_scale_pure_quantum_power_example_1}:
        \item Example~\ref{time_scale_pure_quantum_power_example_2}:
        \item Corollary~\ref{time_scale_pure_quantum_power_corollary_1}:
        \item Corollary~\ref{time_scale_pure_quantum_power_corollary_2}:
    \end{itemize}
\end{itemize}
