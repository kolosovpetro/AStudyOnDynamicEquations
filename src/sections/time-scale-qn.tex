\begin{cor}
    \label{q_derivative_case}
    (Q-derivative~\cite{jackson_1909}.)
    Let be a two-dimensional time scale
    $\Lambda^2 = q^{\mathbb{R}} \times q^{\mathbb{R}} \colonequals \{t=(x,b) \colon \; x\in q^{\mathbb{R}}, \; b \in q^{\mathbb{R}} \}$.
    For every $t\in q^{\mathbb{R}}$ and $(x,b)\in \Lambda^2$
    \begin{align*}
        \qderivative{x^{2m+1}}(t)
        = \pTsDerivative{\polynomialP{m}{b}{x}}{x} (m, \sigma(t), t)
        + \pTsDerivative{\polynomialP{m}{b}{x}}{b} (m, t, t)
    \end{align*}
    where $\sigma(t) = qt, \; q > 1$.
\end{cor}
\begin{examp}
    \label{time_scale_qn_example_1}
    Let be $t\in q^{\mathbb{R}}, \; x,b\in \Lambda^2 = q^{\mathbb{R}} \times q^{\mathbb{R}}$, let $m=1$ then
    \begin{align*}
        &\pTsDerivative{\polynomialP{1}{b}{x}}{x} = -3 b + 3 b^2 \\
        &\pTsDerivative{\polynomialP{1}{b}{x}}{b} = 3 b - 2 b^2 + 3 b q - 2 b^2 q - 2 b^2 q^2 - 3 x + 3 b x + 3 b q x
    \end{align*}
    Evaluating in points yields
    \begin{align*}
        &\pTsDerivative{\polynomialP{1}{b}{x}}{x} (m, \sigma(t), t) = -3 q t + 3 q^2 t^2 \\
        &\pTsDerivative{\polynomialP{1}{b}{x}}{b} (m, t, t) = 3 q t + t^2 + q t^2 - 2 q^2 t^2
    \end{align*}
    Summing up previously obtained partial time-scale derivatives, we get the $q$-derivative of odd-powered polynomial
    $x^{3}$ evaluated in point $t\in q^{\mathbb{R}}$
    \begin{align*}
        \qderivative{x^{3}}(t)
        = \pTsDerivative{\polynomialP{1}{b}{x}}{x} (m, \sigma(t), t)
        + \pTsDerivative{\polynomialP{1}{b}{x}}{b} (m, t, t)
        = t^2 + q t^2 + q^2 t^2.
    \end{align*}
\end{examp}
For every $t\in q^{\mathbb{R}}, \; (x,b) \in \Lambda^2 = q^{\mathbb{R}} \times q^{\mathbb{R}}$
the following polynomial identity holds as $q$ tends to zero
\begin{align*}
    \lim \limits_{q \to 0} \pTsDerivative{\polynomialP{1}{b}{x}}{b} (1, t, t) = t^2
\end{align*}
However, it would be generalized as follows
\begin{cor}
    \label{time_scale_qn_corollary_1}
    For every $t\in q^{\mathbb{R}}, \; (x,b) \in \Lambda^2 = q^{\mathbb{R}} \times q^{\mathbb{R}}$
    \begin{align*}
        \lim \limits_{q \to 0} \pTsDerivative{\polynomialP{m}{b}{x}}{b} (m, t, t) = t^{2m}.
    \end{align*}
\end{cor}
\begin{examp}
    \label{time_scale_qn_example_2}
    Let be $t\in q^{\mathbb{R}}, \; (x,b) \in \Lambda^2 = q^{\mathbb{R}} \times q^{\mathbb{R}}$, let $m=2$ then
    \begin{align*}
        \pTsDerivative{\polynomialP{2}{b}{x}}{x}
        &= -15 b^2 + 30 b^3 - 15 b^4 + 5 b x - 15 b^2 x + 10 b^3 x + 5 b q x - 15 b^2 q x + 10 b^3 q x \\
        \pTsDerivative{\polynomialP{2}{b}{x}}{b}
        &= 10 b^2 - 15 b^3 + 6 b^4 + 10 b^2 q - 15 b^3 q + 6 b^4 q
        + 10 b^2 q^2 - 15 b^3 q^2 + 6 b^4 q^2 - 15 b^3 q^3 \\
        &+ 6 b^4 q^3 + 6 b^4 q^4 - 15 b x + 30 b^2 x - 15 b^3 x - 15 b q x + 30 b^2 q x
        - 15 b^3 q x + 30 b^2 q^2 x \\
        &- 15 b^3 q^2 x - 15 b^3 q^3 x + 5 x^2 - 15 b x^2 + 10 b^2 x^2 - 15 b q x^2 + 10 b^2 q x^2 + 10 b^2 q^2 x^2
    \end{align*}
    Evaluating in points yields
    \begin{align*}
        &\pTsDerivative{\polynomialP{2}{b}{x}}{x} (2, \sigma(t), t)
        = 5 q t^2 - 10 q^2 t^2 - 15 q^2 t^3 + 15 q^3 t^3 + 10 q^3 t^4 - 5 q^4 t^4 \\
        &\pTsDerivative{\polynomialP{2}{b}{x}}{b} (2, t, t)
        = -5 q t^2 + 10 q^2 t^2 + 15 q^2 t^3 - 15 q^3 t^3 + t^4 + q t^4 + q^2 t^4 - 9 q^3 t^4 + 6 q^4 t^4
    \end{align*}
    Summing up previously obtained partial time-scale derivatives, we get the $q-$derivative of odd polynomial
    $x^{5}$ evaluated in point $t\in q^{\mathbb{R}}$
    \begin{align*}
        \qderivative{t^{5}}
        = \pTsDerivative{\polynomialP{2}{b}{x}}{x} (2, \sigma(t), t)
        + \pTsDerivative{\polynomialP{2}{b}{x}}{b} (2, t, t)
        = t^4 + q t^4 + q^2 t^4 + q^3 t^4 + q^4 t^4.
    \end{align*}
\end{examp}
