\begin{cor}
    \label{derivative_case}
    (Classical derivative.)
    Let be a two-dimensional timescale
    $\Lambda^2 = \mathbb{R} \times \mathbb{R} \colonequals \{t=(x, b) \colon \; x\in \mathbb{R}, \; b\in\mathbb{R} \}$.
    For every $t\in\mathbb{R}$ and $(x,b) \in \Lambda^2$
    \begin{align*}
        \odv{x^{2m+1}}{x} (t)
        = \pdv{\polynomialP{m}{b}{x}}{x} (m, \sigma(t), t)
        + \pdv{\polynomialP{m}{b}{x}}{b} (m, t, t)
    \end{align*}
    where $\sigma(t) = t + \Delta t, \; \Delta t \to 0.$
\end{cor}
\begin{examp}
    \label{time_scale_r_example_1}
    Let be $t\in\mathbb{R}, \; (x,b) \in \Lambda^2 = \mathbb{R} \times \mathbb{R}$, let $m=1$ then
    \begin{align*}
        \pdv{\polynomialP{1}{b}{x}}{x} &= -3 b + 3 b^2 \\
        \pdv{\polynomialP{1}{b}{x}}{b} &= 6 b - 6 b^2 - 3 x + 6 b x
    \end{align*}
    Evaluating in points yields
    \begin{align*}
        &\pdv{\polynomialP{1}{b}{x}}{x} (1, \sigma(t), t) = -3t + 3t^2 \\
        &\pdv{\polynomialP{1}{b}{x}}{b} (1, t, t) = 3t
    \end{align*}
    Summing up previously obtained partial timescale derivatives, we get an ordinary derivative of odd polynomial
    $x^{3}$ evaluated in point $t \in \mathbb{R}$.
    \begin{align*}
        \odv{x^3}{x} (t)
        = \pdv{\polynomialP{1}{b}{x}}{x} (1, \sigma(t), t)
        + \pdv{\polynomialP{1}{b}{x}}{b} (1, t, t)
        = 3t^2.
    \end{align*}
\end{examp}
\begin{examp}
    \label{time_scale_r_example_2}
    Let be $t\in\mathbb{R}, \; (x,b) \in \Lambda^2 = \mathbb{R} \times \mathbb{R}$, let $m=2$ then
    \begin{align*}
        &\pdv{\polynomialP{2}{b}{x}}{x} = -15 b^2 + 30 b^3 - 15 b^4 + 10 b x - 30 b^2 x + 20 b^3 x, \\
        &\pdv{\polynomialP{2}{b}{x}}{b} = 30 b^2 - 60 b^3 + 30 b^4 - 30 b x + 90 b^2 x - 60 b^3 x + 5 x^2 - 30 b x^2 + 30 b^2 x^2
    \end{align*}
    Evaluation in points yields
    \begin{align*}
        &\pdv{\polynomialP{2}{b}{x}}{x} (2, \sigma(t), t) = -5 t^2 + 5 t^4 \\
        &\pdv{\polynomialP{2}{b}{x}}{b} (2, \sigma(t), t)        = 5 t^2
    \end{align*}
    Summing up previously obtained partial timescale derivatives, we get classical derivative of an odd polynomial
    $x^5$ evaluated in point $t\in\mathbb{R}$
    \begin{align*}
        \odv{x^5}{x} (t)
        = \pdv{\polynomialP{2}{b}{x}}{x} (2, \sigma(t), t)
        + \pdv{\polynomialP{2}{b}{x}}{b} (2, \sigma(t), t)
        = 5t^4.
    \end{align*}
\end{examp}
