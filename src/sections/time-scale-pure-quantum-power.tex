In this subsection we discuss a pure quantum power time scale $q^{\mathbb{R}^j}$ provided by Aldwoah, Malinowska
and Torres in~\cite{aldwoah2011power}, among with the $q-$power derivative operator $\nqderivative{f(t)}$ defined by
\begin{align*}
    \nqderivative{f(t)} = \frac{f(qt^n) - f(t)}{qt^n - t},
\end{align*}
where $n$ is odd positive integer and $0 < q < 1$.
\begin{cor}
(Quantum power derivative~\cite{aldwoah2011power}.)
    Let be a two-dimensional time scale
    $\Lambda^2 = q^{\mathbb{R}^j} \times q^{\mathbb{R}^j}
    \colonequals \{t=(x,b) \colon \; b \in q^{\mathbb{R}^j}, \; x \in q^{\mathbb{R}^j} \}$.
    For every $t\in q^{\mathbb{R}^j}, \; (x,b) \in \Lambda^2 = q^{\mathbb{R}^j} \times q^{\mathbb{R}^j}$
    \begin{align*}
        \nqderivative{x^{2m+1}} (t)
        = \pTsDerivative{\polynomialP{m}{b}{x}}{x} (m, \sigma(t), t)
        + \pTsDerivative{\polynomialP{m}{b}{x}}{b} (m, t, t)
    \end{align*}
    where $\sigma(t) = qt^n, \; \sigma(t) > t$.
\end{cor}
\begin{examp}
    \label{time_scale_pure_quantum_power_example_1}
    Let be $t\in q^{\mathbb{R}^j}, \; (x,b) \in\Lambda^2 = q^{\mathbb{R}^j} \times q^{\mathbb{R}^j}$, let $m=1$ then
    \begin{align*}
        &\pTsDerivative{\polynomialP{1}{b}{x}}{x} = -3 b + 3 b^2 \\
        &\pTsDerivative{\polynomialP{1}{b}{x}}{b} = 3 b - 2 b^2 + 3 b^j q - 2 b^{1 + j} q - 2 b^{2 j} q^2 - 3 x + 3 b x + 3 b^j q x
    \end{align*}
    Evaluating in points yields
    \begin{align*}
        &\pTsDerivative{\polynomialP{1}{b}{x}}{x}(1, \sigma(t), t) = -3 q t^j + 3 q^2 t^{2 j} \\
        &\pTsDerivative{\polynomialP{1}{b}{x}}{b}(1, t, t) = t^2 + 3 q t^j - 2 q^2 t^{2 j} + q t^{1 + j}
    \end{align*}
    Summing up previously obtained partial timescale derivatives, we get $q-$power derivative of odd-powered polynomial
    $x^{3}$ evaluated in point $t\in q^{\mathbb{R}^j}$
    \begin{align*}
        \nqderivative{x^{3}}(t)
        = \pTsDerivative{\polynomialP{1}{b}{x}}{x} (1, \sigma(t), t)
        + \pTsDerivative{\polynomialP{1}{b}{x}}{b} (1, t, t)
        = t^2 + q^2 t^{2 j} + q t^{1 + j}.
    \end{align*}
\end{examp}

Another polynomial identity, that is exponential sum holds
\begin{cor}
    \label{time_scale_pure_quantum_power_corollary_1}
    For every $t\in q^{\mathbb{R}^j}, \; (x,b) \in\Lambda^2 = q^{\mathbb{R}^j} \times q^{\mathbb{R}^j}, \; t\in\mathbb{R}$
    \begin{align*}
        \lim \limits_{j \to 0} \lim \limits_{q \to 1} \pTsDerivative{\polynomialP{m}{b}{x}}{b} (m, t, t)
        = \sum_{k=0}^{2m} t^k
    \end{align*}
\end{cor}

An identity in even polynomials holds too
\begin{cor}
    \label{time_scale_pure_quantum_power_corollary_2}
    For every $t\in q^{\mathbb{R}^j}, \; (x,b) \in\Lambda^2 = q^{\mathbb{R}^j} \times q^{\mathbb{R}^j}$
    \begin{align*}
        \lim \limits_{j \to 0} \lim \limits_{q \to 0} \pTsDerivative{\polynomialP{m}{b}{x}}{b} (m, t, t)
        = t^{2m}
    \end{align*}
\end{cor}

\begin{examp}
    \label{time_scale_pure_quantum_power_example_2}
    Let be $t\in q^{\mathbb{R}^j}, \; (x,b) \in\Lambda^2 = q^{\mathbb{R}^j} \times q^{\mathbb{R}^j}$, let $m=2$ then
    \begin{align*}
        \pTsDerivative{\polynomialP{2}{b}{x}}{x}
        &= -15 b^2 + 30 b^3 - 15 b^4 + 5 b x - 15 b^2 x + 10 b^3 x + 5 b q x^j - 15 b^2 q x^j + 10 b^3 q x^j \\
        \pTsDerivative{\polynomialP{2}{b}{x}}{b}
        &= 10 b^2 - 15 b^3 + 6 b^4 + 10 b^{1 + j} q - 15 b^{2 + j} q + 6 b^{3 + j} q + 10 b^{2 j} q^2 - 15 b^{1 + 2 j} q^2 \\
        &+ 6 b^{2 + 2 j} q^2 - 15 b^{3 j} q^3 + 6 b^{1 + 3 j} q^3 + 6 b^{4 j} q^4 - 15 b x + 30 b^2 x - 15 b^3 x - 15 b^j q x \\
        &+ 30 b^{1 + j} q x - 15 b^{2 + j} q x + 30 b^{2 j} q^2 x - 15 b^{1 + 2 j} q^2 x - 15 b^{3 j} q^3 x + 5 x^2 - 15 b x^2 \\
        &+ 10 b^2 x^2 - 15 b^j q x^2 + 10 b^{1 + j} q x^2 + 10 b^{2 j} q^2 x^2
    \end{align*}
    Evaluation in points yields
    \begin{align*}
        &\pTsDerivative{\polynomialP{2}{b}{x}}{x} (2, \sigma(t), t)
        = -10 q^2 t^{2 j} + 15 q^3 t^{3 j} - 5 q^4 t^{4 j} + 5 q t^{1 + j} - 15 q^2 t^{1 + 2 j} + 10 q^3 t^{1 + 3 j} \\
        &\pTsDerivative{\polynomialP{2}{b}{x}}{b} (2, t, t)
        = t^4 + 10 q^2 t^{2 j} - 15 q^3 t^{3 j} + 6 q^4 t^{4 j} - 5 q t^{1 + j} + q t^{3 + j} + 15 q^2 t^{1 + 2 j}
        + q^2 t^{2 + 2 j} - 9 q^3 t^{1 + 3 j}
    \end{align*}
    Summing up previously obtained partial timescale derivatives, we $q-$power derivative of odd polynomial
    $x^{5}$ evaluated in point $t\in q^{\mathbb{R}^j}$
    \begin{align*}
        \nqderivative{x^{5}} (t)
        = \pTsDerivative{\polynomialP{1}{b}{x}}{x} (2, \sigma(t), t)
        + \pTsDerivative{\polynomialP{1}{b}{x}}{b} (2, t, t)
        = t^4 + q^4 t^{4 j} + q t^{3 + j} + q^2 t^{2 + 2 j} + q^3 t^{1 + 3 j}
    \end{align*}
\end{examp}
